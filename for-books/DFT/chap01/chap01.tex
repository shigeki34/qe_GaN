\chapter{第一原理計算に取り組む動機}

\begin{summary}
GaNは高い破壊電界、高速な電子速度をもつことから、高周波デバイスとして広く使われており、最近では次世代パワーデバイスの開発も進んでいる。
GaNデバイスの性能向上には、結晶性、絶縁膜や金属との界面、プロセス中の表面状態など、様々な要素を改善しなくてはならない。
なかでも電気特性に大きな影響を与える電子トラップの制御が最も大きな課題の一つである。電子トラップは不純物や空孔といった結晶欠陥が原因であるが、GaNの結晶成長技術もまだ改善途上にあり詳細に調べられておらず、第一原理計算による物性予測が一歩先を行く状態である。
この章では、まずGaNの点欠陥に関する第一原理計算を行った論文とそれをもとに結晶成長条件の探索を行った論文をまとめ、第一原理計算結果の結晶成長への応用について考察する。
\end{summary}

\section{第一原理計算の結晶成長への応用}
\subsection{GaN HEMT開発における課題 電子トラップの抑制}
GaN HEMT開発における課題はいくつかあるが、その中の一つに点欠陥の抑制があげられる。
GaN中の電子トラップによって、電圧印可後に電流が減少し、時間とともに回復する現象が生じる。
高周波デバイスではこの過渡応答が信号のノイズにつながり、EVM (Error Vector Magnitude) が大きくなる。
過渡応答を抑制するためには電子トラップの抑制が不可欠だが、いったい何が原因でこの準位が形成されているのだろうか。

電子トラップの起源を調べるために数多くの実験が行われている。
様々な手法を用いて調査されており、例えばPLやCL測定で評価したGaN中のトラップ準位とその起源は次の文献にまとめられている\cite{defects_gan}。

多くは不純物元素によるものだが、なかには空孔が原因のトラップも存在しており、第一原理計算によってそのエネルギー準位が計算されている\cite{dft_gan_defect_2004},\cite{dft_gan_defect_2017}。
もちろん、不純物によってつくられるエネルギー準位も第一原理計算で計算されており、実験結果を支持するデータとして使われている。

第一原理計算は形成される欠陥の種類や欠陥準位の予測に活用されており、その計算結果を結晶成長へも応用できると考えられる。

\subsection{結晶成長への応用例}
不純物による点欠陥を抑制するための成長条件を探索するために第一原理計算を活用したという報告がある\cite{dft_gan_growth_carbon}, \cite{dft_gan_growth_oxygen}。
Reddyらは、成長中の化学ポテンシャルを変化させることでGaN中の炭素不純物を制御できたと報告し\cite{dft_gan_growth_carbon}、SzymanskiらはN極性GaN中の酸素不純物を低減したと報告している\cite{dft_gan_growth_oxygen}。

彼らは第一原理計算による点欠陥形成エネルギーと結晶成長の熱力学を結びつけるパラメータとして化学ポテンシャルに注目している。
第一原理計算での欠陥形成エネルギーの計算式に化学ポテンシャルの項が存在しており、欠陥の形成されやすさは欠陥に関係する原子の化学ポテンシャルに依存している。
成長中の各原子の化学ポテンシャルは熱力学にもとづいて炉内の原料ガスの分圧から計算することができる。
したがって、化学ポテンシャルを調整弁として点欠陥を抑制できるように成長条件を変更することが可能となる。

彼らの研究は不純物によるものだが、空孔にも応用可能と考えられる。
そこで、彼らの手法をより詳細にまとめたのち、窒素空孔抑制のための成長条件を検討する。

\subsection{第一原理計算による欠陥形成エネルギーの算出}
第一原理計算で点欠陥の形成エネルギーを計算する際には、100原子程度で構成されるスーパーセルを用いた計算がよくおこなわれている。
電荷$q$点欠陥$D$の形成エネルギーを計算する式は、
\begin{equation}
  E_{f}[D^{q}] = \qty{ E[D^{q}] + E_{\rm{corr}}[D^{q}]} - E_{P} - \sum_{i}n_{i}\mu _{i} + q\qty{ \epsilon _{\rm{VBM}}+\Delta \epsilon _{F}}
\end{equation}
とあらわされる。
ここで、$E[D^{q}]$は点欠陥を含むスーパーセルの全エネルギー, $E_{P}$は同じサイズの点欠陥を含まない完全結晶の全エネルギー, $n_{i}$は点欠陥を導入する際に追加した元素$i$の原子数(取り除いた場合は負の数になる), $\mu _{i}$は化学ポテンシャル, $\epsilon _{\rm{VBM}}$は価電子帯上端のエネルギー位置, $\Delta \epsilon _{F}$は価電子帯からのフェルミ準位のエネルギー位置を表す。
点欠陥の濃度は$10^{16} ~ 10^{20}$ cm$^{-3}$程度で物性に影響を与える。しかし、第一原理計算でこのような希薄な点欠陥を計算することは計算コストの点から不可能である。そこで、希薄極限との差分を補正するための項が$ E_{\rm{corr}}[D^{q}]$である。

第一原理計算によって計算した形成エネルギーと実際の成長条件を結び付けるにあたって、化学ポテンシャルの項に注目する。
例えば、窒素原子の位置に炭素原子が入る場合、窒素が一つ抜けて炭素が一つ結晶にはいるため、炭素と窒素の化学ポテンシャルが影響する。
窒素の化学ポテンシャルが高いと窒素が固体から抜けづらく、形成エネルギーが増加する。
Reddyらは、この予想通りに窒素の化学ポテンシャルを上げると、炭素濃度が減少すると報告している。

では、窒素空孔の場合はどうだろうか。
数式を書くと、
\begin{equation}
  E_{f}[V_{N}^{q}] = \qty{ E[V_{N}^{q}] + E_{\rm{corr}}[V_{N}^{q}]} - E_{P} + \mu _{N} + q\qty{ \epsilon _{\rm{VBM}}+\Delta \epsilon _{F}}
\end{equation}
となる。
窒素空孔も窒素が固体から抜けづらくすればよいので、窒素の化学ポテンシャルを上げることで形成エネルギーがあがり、窒素空孔が減少すると考えられる。

\subsection{化学ポテンシャルと成長条件の関係}
先ほどの節でみたように化学ポテンシャルによって点欠陥の形成エネルギーが変化することから、成長条件によって化学ポテンシャルを制御すれば点欠陥の形成を制御できると考えられる。

化学ポテンシャルは原料ガスの分圧から求められる。
結晶成長中のGaN表面におけるGaの化学ポテンシャルはGaの標準状態(金属Ga)と等しく、
\begin{equation}
  \mu_{\rm{Ga}} = kT \log \qty(\frac{P_{eq}^{\rm{Ga}}}{P_{V}^{\rm{Ga}}})
\end{equation}
とあらわせる。
ここで、$P_{eq}^{\rm{Ga}}$は成長中のGaN表面におけるGaの平衡分圧, $P_{V}^{\rm{Ga}}$は成長温度における金属Gaの平行蒸気圧をあらわす。
したがって、成長条件に対してGaの平衡分圧を計算することができれば、化学ポテンシャルを計算することが可能となる。

さらに、GaとNの平衡状態における化学ポテンシャルは次のような関係をもつ。
\begin{align}
  \mu_{\rm{Ga}} + \mu_{\rm{N}} = \mu_{\rm{GaN}} \\
  \mu_{\rm{GaN}} = \mu_{\rm{Ga}_{l}} + 0.5  \mu_{\rm{N}_{2}} + \Delta H_{f}
\end{align}
ここで、$\mu_{\rm{GaN}}$と$\Delta H_{f}$はGaNの化学ポテンシャルと生成エンタルピーをあらわす。
また、基準とする状態からのGaの化学ポテンシャルの変化、$\Delta \mu_{\rm{Ga}}$はNの化学ポテンシャルの変化と関係があり、

となる。
したがって、ある成長条件におけるGaの平衡分圧をもとめれば、Gaの化学ポテンシャルの変化と同時にNの化学ポテンシャルの変化も計算することが可能となる。
\begin{align}
  \Delta \mu_{\rm{Ga}} + \Delta \mu_{\rm{N}} = 0 \\
  \Delta \mu_{\rm{N}} = - \Delta \mu_{\rm{Ga}}
\end{align}
\subsection{GaN MOCVD成長における平衡分圧の計算}
GaN MOCVD成長時の原料の平衡分圧は熱力学を使って計算することができる。
詳細は纐纈らによって報告されており\cite{movpe_gan}, \cite{movpe_gan_jap}、ここではGaNの成長について紹介する。

平衡分圧を計算するにあたって、いくつかの仮定をおこなう。
まず最初の仮定は、MOCVD成長の成長速度はIII族原料の物質輸送律速であるという仮定である。
この仮定のもとでは化学反応速度は原料の到達速度よりも早いため、気相-固相の間で化学平衡が成り立つと仮定することができる。

また、トリメチルガリウム(TMGa)とNH$_{3}$を原料とするGaNの成長を考え、供給された有機金属原料は気相-固相界面で不可逆的に分解し、
\begin{equation}
  \rm{Ga(CH}_{3}\rm{)}_{3}(\rm{g}) + \frac{3}{2}\rm{H}_{2}(\rm{g}) \rightarrow \rm{Ga}(\rm{g}) + 3\rm{CH}_{4}(\rm{g})
\end{equation}
となっていると仮定する。

GaNの成長に関与する分子種はGa, NH$_{3}$, H$_{2}$, 炭化水素(CH$_{4}$), 不活性ガス(N$_{2}$)である。
気相-固相界面におけるGaNの成長反応は、
\begin{equation}
  \rm{Ga}(\rm{g}) + \rm{NH}_{3}(\rm{g}) = \rm{GaN}(\rm{s}) + \frac{3}{2}\rm{H}_{2}(\rm{g})
\end{equation}
とあらわされる。
この式に対する質量作用の法則より、
\begin{equation}
  K = \frac{a_{\rm{GaN}}P_{\rm{H}_{2}}^{3/2}}{P_{\rm{Ga}}\cdot P_{\rm{NH}_{3}}}
\end{equation}
が成り立つ。
また、系の全圧から、
\begin{equation}
  \sum P = P_{\rm{Ga}}+P_{\rm{NH}_{3}} + P_{\rm{H}_{2}} + P_{\rm{N}_{2}} + P_{\rm{CH}_{4}}
\end{equation}
となり、系の束縛条件から、
\begin{align}
  P_{\rm{Ga}}^{in}- P_{\rm{Ga}} &= P_{\rm{NH}_{3}}^{in} - P_{\rm{NH}_{3}} \\
  P_{\rm{CH}_{4}} &= 3 P_{\rm{Ga}} \\
  F &= P_{\rm{H}_{2}} / (P_{\rm{H}_{2}} + P_{\rm{N}_{2}})
\end{align}
となる。ここで、$P_{i}^{in}$は$i$の供給分圧を示し、$P_{i}$は平衡分圧を示す。

NH$_{3}$は300℃以上の温度では熱分解するが、触媒のない通常の成長条件のもとでは分解速度は遅い。そのため、NH$_{3}$の分解率$\alpha$を導入し、成長表面に到達するNH$_{3}$を計算する。
\begin{equation}
  \rm{NH}_{3}(\rm{g}) \rightarrow (1-\alpha ) \rm{NH}_{3}(\rm{g}) + \frac{\alpha}{2} \rm{N}_{2}(\rm{g}) + \frac{3\alpha}{2} \rm{H}_{2}(\rm{g})
\end{equation}

以上の式(1.10)~(1.14)を連立させて解くことで、各原料の平衡分圧を計算することが可能であり、その計算結果を活用して化学ポテンシャルの変化を予想することができる。

\subsection{まとめ}
第一原理計算から点欠陥の形成エネルギーを求めることが可能で、そのエネルギーは欠陥形成にかかわる原子の化学ポテンシャルに依存する。
成長条件における化学ポテンシャルは熱力学に基づいて計算することが可能である。
したがって、熱力学を使って計算した化学ポテンシャルの値を第一原理計算に活用することで、点欠陥の形成を抑制できる成長条件の探索が可能になると考えられる。
次章から、第一原理計算のやり方についてまとめていく。
